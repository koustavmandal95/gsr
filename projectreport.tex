\documentclass[a4paper,12pt]{article}
\usepackage{color} 
\usepackage{graphicx}
\begin{document}
	\begin{center}
		\begin{figure}[h!]
			\centering
			\includegraphics[width=0.3\textwidth]{pes.jpg}
		\end{figure}

		 (\large Established under Karnataka Act No. 16 of 2013) \newline
		   \vspace{8mm}
		100-ft Ring Road, Bengaluru – 560 085, Karnataka, India
		  \vspace{8mm}
		Dissertation on\newline
		 \vspace{8mm}
		\huge Garbage surveillance Robot   \vspace{2mm}\newline
	 \large	Submitted by\newline
	
\large	Aashirwad N. B. (01FB14EEE004)\newline
		Kavin Nishanth (01FB14EEE034) \newline
		\vspace{0.5mm}
		Koustav Mandal (01FB14EEE038) \newline
		       Jan. - Apr.  2018\newline
		     under the guidance of\newline
		     
External Guide  \hfill                              Internal Guide\newline     
Dr. / Prof. XXXXX   \hfill                          Dr. Venkatarangan M. J.\newline
Affiliation	 \hfill  
Departmentof EEE\newline
XXXXXX\hfill Pes University \newline
xxxxxxx\hfill Bengaluru -560085
\end{center} 
\newpage
    \begin{figure}[h!]
	\centering
	\includegraphics[width=0.3\textwidth]{pes.jpg}
    \end{figure} 

    \begin{center}
    \large	FACULTY OF\newline 
	DEPARTMENT OF \newline
		\vspace{8mm}
	PROGRAM \newline
     \Huge CERTIFICATE\newline
      \vspace{4mm}
   \large This is to certify that the Dissertation entitled\newline
       \vspace{2mm}
  \Large GARBAGE SURVEILLANCE \newline 
       ROBOT\newline
  \small is a bonafide work carried out by\newline
  \large
	Aashirwad N. B. (01FB14EEE004) \newline
	Kavin Nishanth (01FB14EEE034)\newline
	Koustav Mandal (01FB14EEE038)\newline
   \end{center}

In partial fulfillment for the completion of VIII semester course work in the Program of Study B Tech in xxxxxxxxx under rules and regulations of PES University, Bengaluru during the period Jan. 2016 – Apr.  2016. It is certified that all corrections/suggestions indicated for internal assessment have been incorporated in the report. The dissertation has been approved as it satisfies the xth semester academic requirements in respect of project work.\vspace{2mm} \newline 

\tiny
Signature with date and Seal  \hspace{1cm}    
Signature with date and Seal   \hspace{1cm}	
Signature with date and Seal

\hspace{1cm}Internal Guide \hspace{2.4cm}
 Chairperson \hspace{2.8cm}	
 \vspace{20mm}
  Dean of Faculty  
 \begin{flushleft}
 \normalsize	Name of the students :-
  \hspace{20cm}	Aashirwad N. B. (01FB14EEE004)
  \hspace{20cm}	Kavin Nishanth (01FB14EEE034)\newline
  	Koustav Mandal (01FB14EEE038)
 	
 \end{flushleft}
 		     \newpage
 		   
 		     \begin{center}
 		     \Huge	  DECLARATION
 		     \end{center}
 \Large
I, xxxxxxx, hereby declare that the dissertation entitled, ‘Title of the work’, is an original work done by us under the guidance of Dr./Prof./Mr./Ms. xxxx, Designation, Affiliation, and is being submitted in partial fulfillment of the requirements for completion of VIIIthSemester course work in the Program of Study B.Tech in xxxxxxx.   
	\vspace{3mm}
\begin{flushleft}
	\vspace{7mm}
\normalsize PLACE:\newline
\vspace{10mm}
DATE: \newline
NAME AND SIGNATURE OF THE CANDIDATE\newline
Aashirwad N. B. (01FB14EEE004) \newline
Kavin Nishanth (01FB14EEE034)\newline
Koustav Mandal (01FB14EEE038)\newline
\end{flushleft}
\newpage
\section{Acknowledgement}
{
The satisfaction and euphoria that accompany the successful completion of any task would be in complete without the mention of the people who made it possible ,whose constant guidance and encourangement crowned the efforts with success.

I would like to profoundly thank the management of Pes University for providing good facilities  for successful completion of the project.

I would like to express my deepest sense of gratitude to my project guide DR. Venkatrangan M. J of Dept. of electrical and electronics  for his constant support throughoout the project.

 
}
\newpage
\tableofcontents


Index\index{keyword}
\section{Abstract}
 {
The main objective of the project is to detect garbage sites in Pes University and alert the house-keeping authorities.The main feature of our robot is an onboard video camera.
The main control unit of this robot is raspberry pi 3 with wifi module and a remote control .

 }
\section{Introduction}
This project deals with classification of garbage into various caterories using the software tensor.
Along with a learn and repeat system. Upon detection of the garbage it will send a signal to the house-keeping authorities. Hence the main feature of our robot is an onboard video camera. Also the robot must be compact and self contained in the sense it must have an onboard battery pack and wireless interface to the human controller.

\section{chapter}
{
	\subsection{specifications}
	{
		
		User requirement:\newline
		Should \newline
		
		1. Survey a distance of 200m in about 15 min \newline
		
		2. Complete at least 4 round-trips before being recharged \newline
		
		Technical requirement:\newline
		
		Distance between the two points = 200m \newline
		Assuming it stops every 3m to capture an image of its surroundings and takes 5s to capture and process the image, the time taken to traverse the path = 4 min\newline
		Or an average speed of approximately 75 cm/s\newline
		
		Robot includes\newline
		4 DC motors with gearboxes = 4 * 180 gm = 720 gm [1] \newline
		4 wheels = 120 gm [2]
		chassis = 250 gm [2]  
		Raspberry Pi 3 model B (with wifi capability) = 45 gm
		Assume total battery weight = 250 gm [3] 
		Miscellaneous = 100 gm
		
		Total = approx 1500 gm \newline
		
		Power and Torque (Detailed calculations with comments in SpecsCalculator.m) \newline
		
		With a maximum angle of inclination = . degrees and rolling friction coefficient = 0.05
		
		Top Speed = 75 cm/s \newline
		Total Torque =  45   \newline
		Motor RPM =  200 RPM  \newline
		Raspberry Pi Battery WH =  8 Wh \newline
		RaspPi Battery =  1600 mAh (at 5.1V) \newline
		Motor Battery WH = 5 Wh \newline
		Motor Battery MAH =  400 mAh \newline
		Total Watt-Hours =  13 \newline
		}
	}




\end{document}
